\documentclass[toc,DM,lsstdraft]{lsstdoc}
\usepackage{color}
\title{LDF File Systems Baseline Overview}
\author{Don Petravick}
\date{2017-08-03}
\setDocRef{DMTN-051}
\setDocUpstreamLocation{\url{https://github.com/lsst-dm/dmtn-051}}

\setDocAbstract{%
This is a descriptive and explanatory document, not a normative
document. This document explains the proposed baseline as presented in
the DM replan in July, 2017, referred to just ``baseline'' in the prose
that follows.
}

\begin{document}
\maketitle

\section{Overview}

The baseline for the LSST provides for two types of file systems --
``site'' file systems, which are local to LDF systems at the Base or at
LDF systems at NCSA, and ``data backbone file systems'' that hold the
permanent record of data facility operations .

In baseline, all file systems described below are Posix file systems. A
subset of these file systems are backed up, as indicated.

\section{Site File Systems}

\begin{longtable}[]{@{}lll@{}}
\hline
\textbf{File System} & \textbf{Where} & \textbf{Backed up?}\tabularnewline
\hline
\endhead
Login & NCSA & Yes\tabularnewline
Project & NCSA & No\tabularnewline
Scratch & NCSA & No\tabularnewline
PDAC-user & NCSA & Yes\tabularnewline
USDAC-User & NCSA & Yes\tabularnewline
Backup & NCSA & n/a\tabularnewline
Datasets & NCSA & Yes\tabularnewline
Kubernetes containers & NCSA & Yes\tabularnewline
Production login & NCSA & Yes\tabularnewline
L1 Alert Log File system & NCSA & No\tabularnewline
L1 Input/Output Cache & NCSA & No\tabularnewline
Template Cache & NCSA & NO\tabularnewline
Login & Base & Yes\tabularnewline
Project & Base & No\tabularnewline
Scratch & Base & No\tabularnewline
Chilean DAC-USER & Base & Yes\tabularnewline
L1 application Support & Base & Yes\tabularnewline
Observatory Operations File System & Base & No\tabularnewline
Datasets & Base & Yes.\tabularnewline
Kubernetes containers & Base & Yes\tabularnewline
Backup & Base & n/a\tabularnewline
\hline
\end{longtable}

Login File systems serve users at each site, and are backed up. Login
file system are backed up.

Project file systems provide a place for output of informal,
development-oriented outputs of running codes. These are not backed up,
on the assumption that data her can be regenerated.

Scratch file systems hold temporary program output files, and are
subject to advertised purge policies. This distinguishes scratch file
systems from Project file systems, which are not purged by system
administrator applying a policy.

Dataset file systems hold designated datasets that supports development.
Designated datasets are backed up, (some datasets are easily obtain-able
from the internet, and do not require a backup. However files in the
dataset file systems are not considered to part of the formal record of
the survey. Any files that would become part of the formal system would
be replicated and kept in the data backbone, discussed below.

The DAC file systems (PDAC, USDAC, CHILEANDAC) hold user-created files
for the corresponding data access centers. The file systems need to be
backed up.

Backup file systems provide for copies of recent data. Backup file
systems have independent implementation from the systems they backup.
Backup file systems have a role as secondary disaster recovery data
recovery resources, but are not primary sources of storage for disaster
recovery.

The template Cache, L1 Input/Output file system, and L1 Alert Log
support pay special role supporting prompt processing at NCSA.

The L1 application support file system support LDF L1 services at the
base.

The Observatory Operations file server is an intermediate file system at
the Base. The spectrograph, Comcam and LSSTCam archivers write this file
system, which is exported to designated observatory computers, including
the commission cluster. Data from this file system is further ingested
into the raw file partition of the e data backbone.

\section{Data Backbone File Systems}

Data backbone file systems have these main distinguishing
characteristics.

\begin{itemize}
\item
  These file systems hold the permanent file-based record of the survey.
\item
  Data on these file systems are synchronized between the base and NCSA
  sites according to policy.
\item
  These file systems are subject to change-controlled disaster recovery
  plans.
\item
  All files are supported by location, meta-data, and provenance
  database entries.
\item
  The integrity of these data residing on these file systems are
  routinely monitored using cataloged meta-data.
\item
  Data are only entered into these file system by processes run in
  change-controlled production processes.
\item
  File systems are presented as POSIX file systems.
\item
  Designated data may only be accessed by access methods which stage
  data from tape.
\item
  Only data from data backbone file systems are used to create formal
  production products.
\item
  Only data from data backbone file system are use to produce file based
  products specified in the DPDD.
\end{itemize}

The baseline provides provides for the following file systems:

The \textbf{Raw} file system holds pixel data from the spectrograph,
COMcam, and LSSTcam and file ingested from the Engineering and Facility
large file annex.

The \textbf{Calibrations} file system holds computed calibration
products. Raw data leading to calibrations, such as raw flat and bias
frames are held in the raw file system.

The \textbf{Production Products} file system contains file other than
calibration file products produced with the formal production system.

The \textbf{Computed Engineering Products} File system holds files
related to the miscellaneous production processes, not covered above.

\end{document}
